\documentclass[11pt]{article}

\usepackage[T1]{fontenc}
\usepackage[polish]{babel}
\usepackage[utf8]{inputenc}
\usepackage{lmodern}
\selectlanguage{polish}

\usepackage{graphics}
\usepackage{graphicx}

\usepackage[dvips]{hyperref}

\author{Michał Bobowski, Marcin Cieślikowski}
\date{2014-01-16}
\title{Narzędzie do analizy statystycznej słów i bigramów.}

\begin{document}
  \maketitle

\section{Wstęp}
Niniejszy dokument stanowi podsumowanie projektu z przedmiotu WEDT.
Zawiera opis interfejsu użytkownika i logiki programu oraz przedstawienie struktury kodu źródłowego.

\section{Koncepcja programu}
Program ma za zadanie obliczyć statystyki słów i bigramów z jednego lub wielu plików wejściowych.
Obsługiwane powinny być najbardziej popularne formaty plików w tym pliki MS Word.
Wyniki obliczeń powinny być przechowywane na dysku w rozsądnym formacie i możliwe do wykorzystania w przyszłości.

Program powinien zawierać prosty graficzny interfejs użytkownika.
Jego zadaniem jest pobranie parametrów obliczeń oraz wyświetlenie wyników symulacji.
Możliwa powinna być również filtracja wyświetlanych danych danych.

\section{Przypadki użycia}
\subsection{Przeprowadzenie obliczeń}
\begin{enumerate}
 \item Wybranie przez użytkownika parametrów obliczeń.
 \item Przeprowadzenie symulacji.
 \item Zapis wyników do pliku.
 \item Wyświetlenie wyników na ekranie.
\end{enumerate}
\subsection{Wczytanie wyników}
\subsection{Filtracja}

\section{Specyfikacja szczegółowa}
\subsection{Parametry wejściowe}
\subsection{Format danych wyjściowych}

\section{Wybór narzędzi i technologii}

\section{Opis kodu źródłowego}

\section{Algorytmy}


\end{document}