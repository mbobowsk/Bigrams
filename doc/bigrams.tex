\documentclass[11pt]{article}

\usepackage[T1]{fontenc}
\usepackage[polish]{babel}
\usepackage[utf8]{inputenc}
\usepackage{lmodern}
\selectlanguage{polish}

\usepackage{graphics}
\usepackage{graphicx}

\usepackage[dvips]{hyperref}

\author{Michał Bobowski, Marcin Cieślikowski}
\date{2014-01-16}
\title{Narzędzie do analizy statystycznej słów i bigramów.}

\begin{document}
  \maketitle

\section{Wstęp}
Niniejszy dokument stanowi podsumowanie projektu z przedmiotu WEDT.
Zawiera opis interfejsu użytkownika i logiki programu oraz przedstawienie struktury kodu źródłowego.

\section{Przypadki użycia}
Uznaliśmy, że określenie przypadków użycia jest najprostszym sposobem zapisania wymagań wstępnych.
Scenariusze można też traktować jako zwięzłą instrukcję obsługi.

\subsection{Przeprowadzenie obliczeń}
Podstawowym zadaniem programu jest przeprowadzenie obliczeń i wyświetlenie ich na ekranie.
Scenariusz tworzą następujące zdarzenia:
\begin{enumerate}
 \item Użytkownik wybiera parametry.
 \item System przeprowadza obliczenia.
 \item System zapisuje wyniki do pliku.
 \item System wyświetla wyniki na ekranie.
\end{enumerate}

\subsection{Wczytanie wyników}
Wysoce prawdopodobnym jest powrót użytkownika do wyników już przeprowadzonych obliczeń.
Mając to na uwadze, oraz biorąc pod uwagę czasochłonność symulacji, wprowadziliśmy możliwość wczytania danych z pliku.
\begin{enumerate}
 \item Użytkownik wybiera ścieżkę do pliku z zapisanymi wynikami.
 \item System wyświetla wyniki na ekranie.
\end{enumerate}

\subsection{Filtracja}
Operacja filtracji staje się dostępna dopiero po wykonaniu któregoś z wcześniejszych przypadków użycia.
Aby nie komplikować wyglądu tabel wynikowych wypełnianie filtrów zostało przeniesione do oddzielnego dialogu.
Uruchomienie okna filtracji jest możliwe przy pomocy menu kontekstowego.

\begin{enumerate}
 \item Użytkownik wybiera jedną z tabel wynikowych.
 \item Użytkownik otwiera dialog filtracji przy użyciu menu kontekstowego.
 \item Użytkownik wypełnia filtry.
 \item System wyświetla przefiltrowane wyniki na ekranie.
\end{enumerate}

\section{Specyfikacja szczegółowa}
W tej części doprecyzowane zostały wymagania dotyczące danych wejściowych i wyjściowych.
\subsection{Parametry wejściowe}
Przed przeprowadzeniem obliczeń użytkownik może zdefiniować następujące parametry:
\begin{itemize}
 \item Ścieżka do pliku wejściowego lub katalogu zawierającego wiele plików wejściowych.
 \item Typ bigramu: obliczany dla kolejnych słów lub wszystkich słów w tekście.
 \item Części mowy dla słów z bigramu.
 \item Nazwa pliku wyjściowego.
\end{itemize}

\subsection{Format danych wyjściowych}
Statystyki słów/bigramów są liczone i prezentowane dwa razy - dla słów z odmianą oraz dla formy podstawowej.
Statystyki dla słów:
\begin{itemize}
 \item Liczba wystąpień w zbiorze.
 \item Liczba zdań w których wystąpiło słowo.
 \item Liczba dokumentów w których wystąpiło słowo.
 \item Procent dokumentów w których wystąpiło słowo.
 \item Miara tf-idf.
\end{itemize}

Statystyki dla bigramów:
\begin{itemize}
 \item Liczba wystąpień w zbiorze.
 \item Liczba zdań w którtch wystąpił bigram.
 \item Liczba dokumentów w których wystąpił bigram.
 \item Procent dokumentów w których wystąpił bigram.
 \item Miara tf-idf.
 \item Prawdopodobieństwo słowa 1.
 \item Prawdopodobieństwo słowa 2.
 \item Prawdopodobieństwo bigramu złożonego ze słów 1 i 2.
\end{itemize}

\section{Wybór narzędzi i technologii}
Program został zrealizowany w języku Java.
Kod tworzyliśmy przy użyciu środowiska Eclipse oraz częściowo Netbeans (edytor interfejsu użytkownika).
Do kontroli kodu wykorzystaliśmy system Git.

Do oznaczenia części mowy wykorzystaliśmy bibliotekę Gate.
Korzysta ona wewnętrznie z biblioteki TIKA, dzięki czemu uzyskaliśmy wsparcie dla wielu formatów tekstowych m. in. txt, html, odt i doc.
Najważniejszym elementem zapożyczonym ze środowiska Gate jest automatyczny POS-tagger dla języka angielskiego.

Dane wyjściowe są przechowywane na dysku w formacie bazy danych SQLite.
Jest to efektywne i uniwersalne rozwiązanie.

\section{Uruchomienie programu}
Do uruchomienia programu niezbędne jest ściągnięcie biblioteki GATE oraz ustawienie zmiennej gate.home.
Zmienna jest argumentem maszyny wirtualnej - w środowisku Eclipse należy wejść w menu Run -> RunConfigurations -> Arguments i tam wpisać np. -Dgate.home="/home/preston/GATE\_Developer\_7.1".

\section{Opis kodu źródłowego}
Kod programu wraz z historią zmian jest dostępny w repozytorium pod adresem https://github.com/mbobowsk/Bigrams .

Kod źródłowy staraliśmy się rozplanować zgodnie z ideą wzorca MVC.
Wyodrębniliśmy cztery pakiety, grupując w nich podobne funkcjonalności.
\subsection{Pakiet view}
Pakiet view zawiera wszystkie klasy związane bezpośrednio z widokiem.
Duża część kodu znajdującego się w tym pakiecie została automatycznie wygenerowana przez edytor formularzy NetBeans.
Główne okno programu reprezentuje klasa \emph{AppWindow}.
Pozostałe klasy obsługują dialogi oraz logikę w nich zawartą (głównie filtrowanie danych).

\subsection{Pakiet model}
Pakiet model jest pokłosiem architektury biblioteki Swing.
Każdy element wyświetlający grupę danych (listy, tabele) potrzebuje powiązanego modelu.
Dodatkowo pakiet zawiera klasę \emph{ModelLogic}, która odpowiada za incjalizację modelu części mowy.

\subsection{Pakiet sql}
Wszelkie operacje związane z bazą danych są wykonywane w pakiecie sql.
Dodatkowo wydzielone zostały oddzielne klasy dla zapisu i odczytu (\emph{SQLWrite} i \emph{SQLRead}).

\subsection{Pakiet logic}
W pakiecie logic znajduje się pozostała częśc logiki oraz typy danych.
Klasa \emph{Controller} stanowi warstwę logiki, do której trafiają dane z widoku.
\emph{Controller} posiada jedną metodę publiczną, która przyjmuje na wejściu opcje programu (obiekt klasy \emph{Options}), a następnie przeprowadza wszystkie obliczenia.
Enkapsuluje ona wywołania biblioteki Gate, poprzez którą przeprowadzany jest podział na tokeny, podział na zdania oraz oznaczanie części mowy.

\section{Algorytm obliczania statystyk}


\end{document}